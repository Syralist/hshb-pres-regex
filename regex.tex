\documentclass[aspectratio=1610,svgnames]{beamer}

\usepackage{lmodern}
\usepackage[T1]{fontenc}
\usepackage[ngerman]{babel}
\usepackage{selinput}
\SelectInputMappings{%
   adieresis={ä},
   germandbls={ß}
   }
\usepackage{csquotes}

\usepackage{hyperref}

\usetheme{PaloAlto}  %% Themenwahl

\setbeamercovered{transparent}
%\setbeamertemplate{footline}[frame number]
\usecolortheme{spruce}        % grün
\usecolortheme[named=MSUgreen]{structure}
\newcommand{\hshblogo}{\includegraphics[width=1.3cm]{space-logo}}
\newcommand{\divider}[1]{\begin{frame} %
\begin{alertblock}{} %
\centering\usebeamerfont{section title}#1 %
\end{alertblock} %
\end{frame}}
 
\newcommand{\code}[1]{\texttt{#1}}
\newcommand{\regex}[1]{\texttt{/#1/}}
\newcommand{\bs}{\textbackslash}

\title{Keine Angst vor Regular Expressions}
\author{Thomas Helmke}
\date{00.00.2018}
\logo{\includegraphics[width=1.1cm]{space-logo}}
 
\begin{document}
\maketitle
\frame{\tableofcontents}

\section{Einleitung}
\divider{\insertsection}
\begin{frame}[fragile]%[<+->] %%Eine Folie
    \frametitle{What sorcery is this?} %%Folientitel
    \begin{itemize}[<+->]
        \item \verb|^([\w-]+(?:\.[\w-]+)*)|\\
            \verb|@((?:[\w-]+\.)*\w[\w-]{0,66})\.([a-z]{2,6}(?:\.[a-z]{2})?)$|\only<4>{\footnote{Emailadresse}}
        \item \verb@^(25[0-5]|2[0-4][0-9]|[01]?[0-9][0-9]?)\.@\\
            \verb@(25[0-5]|2[0-4][0-9]|[01]?[0-9][0-9]?)\.@\\
            \verb@(25[0-5]|2[0-4][0-9]|[01]?[0-9][0-9]?)\.@\\
            \verb@(25[0-5]|2[0-4][0-9]|[01]?[0-9][0-9]?)$@\only<4>{\footnote{IPv4 Adresse}}
        \item \verb|^((?=\S*?[A-Z])(?=\S*?[a-z])(?=\S*?[0-9]).{6,})\S$|\only<4>{\footnote{Passwort, min. 6 Zeichen, 1 Gross, 1 Klein, 1 Ziffer}}
    \end{itemize}
\end{frame}
\begin{frame}%[<+->] %%Eine Folie
    \frametitle{Definition} %%Folientitel
    \begin{exampleblock}{}
    \begin{quote}
        Ein regulärer Ausdruck (englisch regular expression, Abkürzung RegExp oder Regex) 
        ist in der theoretischen Informatik eine Zeichenkette, die der Beschreibung von Mengen 
        von Zeichenketten mit Hilfe bestimmter syntaktischer Regeln dient.\\
        \hspace*\fill{\small--- Wikipedia\footnotemark}
    \end{quote}
    \end{exampleblock}
    \footnotetext{\url{https://de.wikipedia.org/wiki/Regex}}
\end{frame}

\subsection{Zeichenauswahl}
\divider{\insertsubsection}
\begin{frame}[<+->]
    \frametitle{Ein oder mehr Zeichen}
    \begin{itemize}
        \item \regex{thomas} findet \enquote{thomas} als wort
        \item \regex{[thomas]} findet \enquote{t} oder \enquote{h} oder \enquote{o} oder \enquote{m} oder \enquote{a} oder \enquote{s}
        \item \regex{[a-z]} findet einen beliebigen Kleinbuchstaben
        \item \regex{[0-9]} findet eine beliebige Zahl
    \end{itemize}
\end{frame} 
\begin{frame}[<+->]
    \frametitle{Vordefinierte Zeichengruppen}
    \begin{itemize}
        \item \regex{\bs d} findet eine beliebige Zahl (digit)
        \item \regex{\bs w} findet Zahlen, Buchstaben und Unterstriche (word)
        \item \regex{\bs s} findet Leerzeichen, Tabs, Zeilenumbruch etc. (whitespace)
        \item Großbuchstabe funktioniert als Negierung
    \end{itemize}
\end{frame} 
\subsection{Zeichen zählen}
\divider{\insertsubsection}
\begin{frame}[<+->]
    \frametitle{Quantifier}
    \begin{itemize}
        \item \regex{a?} null- oder einmal
        \item \regex{a+} mindestens einmal
        \item \regex{a*} beliebig oft
        \item \regex{a\{n\}} genau n-mal
        \item \regex{a\{n,\}} mindestens n-mal
        \item \regex{a\{n,m\}} mindestens n-mal, maximal m-mal
        \item \regex{a\{0,m\}} maximal m-mal
    \end{itemize}
\end{frame} 
\begin{frame}[<+->]
    \frametitle{Lazy or Greedy?}
    \begin{itemize}
        \item \regex{a*} Greedy Quantifier: findet so viele Zeichen wie möglich
        \item \regex{a*?} Lazy Quantifier: findet so wenig Zeichen wie möglich
    \end{itemize}
\end{frame}

\section{Beispiele}
\divider{\insertsection}
\subsection{Beispiel 1: Längenangabe im Logfile}
\divider{\insertsubsection}
\begin{frame}[<+->]
    \frametitle{Auszug aus Logfile}
    \begin{exampleblock}{}
        \texttt
        21 . 3. 2017  15 : 26 : 32\\
        Abweichung beträgt: {\color<2>{red}-0.23} m
    \end{exampleblock}
\end{frame} 
\begin{frame}[fragile]
    \frametitle{Auszug aus Logfile}
    \begin{itemize}[<+->]
        \item Livedemo auf \url{https://www.regex101.com}
        \item Lösungsansatz: \verb|[-]??\d+[.]\d+|
        \item vollständige Regex: \verb|Abweichung beträgt: ([-]??\d+[.]\d+) m|
    \end{itemize}
\end{frame}
\begin{frame}[fragile]
    \frametitle{Auszug aus Logfile}
    \begin{exampleblock}{Python-Umsetzung}
        python code
    \end{exampleblock}
\end{frame}


\section{Fortgeschrittene Techniken}
\divider{\insertsection}
\subsection{Variablen}
\divider{\insertsubsection}
\begin{frame}[<+->]
    \frametitle{Variablen I}
    \begin{itemize}
        \item Werte können in Variablen gespeichert werden
        \item Variablen können mathematisch verrechnet werden
    \end{itemize}
    \uncover<+->{%
    \code{xsize = 3.5;\newline%
        ysize = xsize;\newline%
        zsize = 0.5*xsize;}}
\end{frame}



\section{Weitere Infos}
\divider{\insertsection}
\begin{frame}[fragile]
    \frametitle{Weitere Infos}
    \begin{itemize}
        \item \url{https://github.com/Syralist/hshb-pres-regex}
        \item \url{https://regex101.com/}
        \item \url{https://regexr.com/}
    \end{itemize}
\end{frame}
\end{document}
